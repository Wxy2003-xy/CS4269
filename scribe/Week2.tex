\documentclass{article}
\usepackage[utf8]{inputenc}    
\usepackage[T1]{fontenc}       
\usepackage{lmodern}           
\usepackage{amsmath}   
\usepackage{amssymb}   
\usepackage{geometry}  
\usepackage{enumerate} 
\usepackage{xcolor}  
\usepackage{amsthm}
\usepackage{pdfpages}
\newtheorem{theorem}{Theorem}[section]
\newtheorem{lemma}[theorem]{Lemma}
\newtheorem{corollary}[theorem]{Corollary}
\newtheorem{definition}[theorem]{Definition}
\newtheorem{example}[theorem]{Example}
\usepackage{listings}  
\usepackage{tikz-cd}
\usepackage{forest}     
\usetikzlibrary{arrows.meta}
\usetikzlibrary{arrows.meta,decorations.pathreplacing,calc}
\lstset{frame=tb,
  language=C,
  aboveskip=3mm,
  belowskip=3mm,
  showstringspaces=false,   
  columns=flexible,
  basicstyle={\small\ttfamily},
  numbers=none,
  numberstyle=\tiny\color{gray},
  keywordstyle=\color{blue},
  commentstyle=\color{brown},
  stringstyle=\color{orange},
  breaklines=true,
  breakatwhitespace=true,
  tabsize=3
}
\geometry{top=1in, bottom=1in, left=1in, right=1in}
% ctrl shift D to toggle night 
\begin{document}

\title{Week 2: Propositional Logic}
\author{Wang Xiyu}
\date{}
\maketitle
Propositional logic is about, well, logics about propositions. A prposition is basically a statement that can be 
evaluated to true or false. For example "1+1=2" and "1+1=3" are both propositions. 
\section{Syntax}
\begin{definition}
    Syntax decides what are allowed to be written
\end{definition}
\subsection{Well-formed Formula (WFF)}
A formula $\psi$ is obtained by composing $p \in P$ with $c\in C$, where $P$ is the set of propositions and $C$ is the set of connectors. here we use
any set of complete logic, like $C = \{\neg, \lor\}$. This is not a percise definition as counter examples that do not form 
a valid formula can be formulated based on this, for example $\neg \lor p$ means nothing. To formalize the definition we may attempt to define a language with the following grammar:
\[\Sigma = P\cup C\cup \{(, )\}\]
\[FORM :\{\psi = p \in P | \neg \psi | \psi \lor \psi | (\psi) \}\]
However this is a recursive definition as the definition of the a formula $\psi$ involves itself. We need to find a declarative way to define this recursive/inductive construction. 

\begin{definition}
    FORM is \textbf{the smallest} set of strings over $\Sigma$ where 
    \begin{enumerate}
        \item $P \subseteq FORM$
        \item $\forall \psi \in FORM, \neg \psi \in FORM$
        \item $\forall \psi_1, \psi_2 \in FORM, \psi_1 \lor \psi_2 \in FORM$
    \end{enumerate}
\end{definition}
We can see as a construction rule, we are admitting strings into the set of formula, but we need to somewhat impose restriction on what is not acceptable. 
\subsection{Closed Set}
Let's remove the attributive \textbf{the smallest} first and define another set that contains all the set that satisfy these rules listed above:
\begin{definition}
    Closed is a set of strings over $\Sigma$ where 
    \begin{enumerate}
        \item $P \subseteq Closed$
        \item $\forall \psi \in Closed, \neg \psi \in Closed$
        \item $\forall \psi_1, \psi_2 \in Closed, \psi_1 \lor \psi_2 \in Closed$
    \end{enumerate}
\end{definition}
\subsubsection{All Closed Set}
And we define a set ACS (All-closed sets). We can define FORM via the following constriction:
\[FORM = \bigcap ACS\]
Suppose $BCS = \emptyset$ and from definition
\[U_0 = \bigcap BCS = \{w \in \Sigma^* | \forall S \in BCS, w \in S\}\]
However, $BCS = \emptyset$, The statement $\forall S \in BCS, w \in S$ is vacuously true $\forall w \in \Sigma^*$, which means that
\[U_0 = \Sigma^* = \bigcap BCS\]
A contradiction.
Therefore we have proved that ACS cannot be empty, thus $\forall P, \exists S$ such that $S$ is a closed set. 
\subsection{FORM is the minimum of the set of Closed}
First let's prove that FORM is closed. It seems to be trivial that the intersection of closed sets is closed but formal proof is still needed.
We denote a set from the intersection construction as $T_0$.
\begin{lemma}
    $T_0$ is closed 
\end{lemma}
We are proving that the set $T_0$ we get from the intersection construction still satisfy the rules that define what is closed. 
\begin{proof}
    From rule 1,
    \begin{align*}
        \forall S &\in ACS, P \subseteq S\\
        &\implies \forall p \in P, \forall S \in ACS, p \in S \\
        &\implies \forall p \in P, p \in \bigcap ACS\\
        &\implies P \in T_0
    \end{align*}
    From rule 2, 
    \begin{align*}
        \forall S &\in ACS, T_0 \subseteq S\\
        \text{let } w &\in T_0, \forall S \in ACS, w \in S, \neg w \in S\\
        & \implies \forall w \in T_0, \neg w \in \bigcap ACS\\
        & \implies \forall w \in T_0, \neg w \in T_0
    \end{align*}
    From rule 3,
    \begin{align*}
        \forall S&\in ACS, T_0 \subseteq S\\
        \text{let } w_1, w_2 &\in T_0, \forall S \in ACS, w_1 \lor w_2 \in S\\
        & \implies \forall w_1, w_2 \in T_0, w_1 \lor w_2 \in \bigcap ACS\\
        & \implies \forall w_1, w_2 \in T_0, w_1 \lor w_2 \in T_0
    \end{align*}
\end{proof}
\begin{lemma}
    $T_0$ is \textbf{a} smallest Closed set
\end{lemma}
\begin{proof}
    The proof of minimality is trivial. $T_0 = \bigcap ACS \implies \forall S \in ACS, T_0 \subseteq S$
\end{proof}
\begin{lemma}
    $T_0$ is \textbf{the} smallest Closed set (unique). In other word, 
    \[\forall T_0' \in \{T | \forall S \in ACS, T \subseteq S\}, T_0' = T_0\]
\end{lemma}
\begin{proof}
Suppose otherwise, $\exists T_0' \in ACS \neq T_0$. 
\begin{align*}
    T_0, T_0' \in ACS &\implies \forall S \in ACS, T_0 \subseteq S, T_0' \subseteq S\\
    &\implies T_0 \subseteq T_0'\land T_0' \subseteq T_0\\
    &\implies T_0 = T_0'
\end{align*}
A contradiction.
\end{proof}
Now we can use the generative rule to define formula, which is called \textbf{Backus-Naur form}. Note that a context-free grammar is a type of formal language. 
Backus Naur form is a specification language for this type of grammar. It is used to describe language syntax.

\section{Sematics}
\begin{definition}
    Sematics describe meanings.
\end{definition}
\subsection{Truth Assignment/Proposition evaluation}
A truth assignment or an evaluation of propositions is a function that maps from the set of propositions to true and false.
\[v: P \rightarrow \{T, F\}\]
\[v \in [P \rightarrow \{T, F\}]\]
\begin{example}
    Consider $v = \{p_1 \mapsto T, p_2 \mapsto F\}$, we can derive
    \begin{align*}
    v&(p_1) = T\\
    v&(p_2)= F\\
    v&(\neg p_2) = T\\
    v&(p_1\lor p_2) = T \text{ etc.}
    \end{align*}
\end{example}
\subsection{Entail/Model}
We denote entailment of an evaluation function to formula as 
\[v \models \psi\]
We also define the following rule 
\begin{align*}
    \forall p \in P, v\models p &\iff v(p)\\
     v\models \psi &\iff v\nvDash \neg \psi\\
     v \models \psi_1 \lor \psi_2 &\iff (v\models \psi_1) \lor (v\models \psi_2)
\end{align*}



\section*{Tutorial}
\subsection{Generic closure}
\begin{definition}
Given a universe set $U$ and an operator set $\mathcal{O}$, where $\forall f \in \mathcal{O}, f$ is a function that mapes , $f$ has arity $ r = arity(f)$. 
\end{definition}
\end{document}