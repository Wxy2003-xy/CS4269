\documentclass{article}
\usepackage[utf8]{inputenc}    % For UTF-8 character encoding
\usepackage[T1]{fontenc}       % For proper font encoding
\usepackage{lmodern}           % Improved font rendering
\usepackage{amsmath}   % For advanced mathematical formatting
\usepackage{amssymb}   % For mathematical symbols
\usepackage{geometry}  % Adjust page margins
\usepackage{enumerate} % For custom lists
\usepackage{xcolor}  % for coloring
\usepackage{amsthm}
\usepackage{pdfpages}
\newtheorem{theorem}{Theorem}[section]
\newtheorem{lemma}[theorem]{Lemma}
\newtheorem{corollary}[theorem]{Corollary}
\newtheorem{definition}[theorem]{Definition}
\usepackage{listings}  % for code listings
\usepackage{tikz-cd}
\usepackage{forest}     
\usetikzlibrary{arrows.meta}
\usetikzlibrary{arrows.meta,decorations.pathreplacing,calc}
\lstset{frame=tb,
  language=C,
  aboveskip=3mm,
  belowskip=3mm,
  showstringspaces=false,   
  columns=flexible,
  basicstyle={\small\ttfamily},
  numbers=none,
  numberstyle=\tiny\color{gray},
  keywordstyle=\color{blue},
  commentstyle=\color{brown},
  stringstyle=\color{orange},
  breaklines=true,
  breakatwhitespace=true,
  tabsize=3
}
\geometry{top=1in, bottom=1in, left=1in, right=1in}

\begin{document}

\title{Pre Quiz}
\author{Wang Xiyu}
\date{}
\maketitle
\section{}
(a), and (b) if "is friends" a reflexive relation.
\section{}
Cannot conclude
\section{}
\[\left(\exists f \in F, f(CS4269)\right) \land (TY \in F)\]
a) $TY(CS4269)$ cannot conclude\\
b) $\forall f \in F, f(CS4269)$ cannot conclude\\
\section{}
\[\forall n \in N, \exists r \in R, r = \sqrt{n}\]
$5.5 \notin N$, cannot conclude $\exists r \in R, r = \sqrt{5.5}$\\
$5 \in N \implies \exists r \in R, r = \sqrt{5}$
\section{}
A set is a collection of unique elements. like $\mathbb{N}$. 
The cardinality of a set describes its size. like $|\mathbb{N}| = \aleph_0$
\[\exists \text{ injection } f: A \mapsto B, |A| \leq |B|\]
\[\exists \text{ surjection } f: A \mapsto B, |A| \geq |B|\]
\section{}
A finite set is a set that contains, well finite amount of element. For example $\emptyset$\\
A countable set is either finite or has the same cardinality as $\mathbb{N}$. for example $\mathbb{Z^+}$\\
An uncountable set is a set that has the same cardinality as $\mathbb{R}$. for example $\mathcal{P}(\mathbb{N})$
\section{}
A function is a mapping from a set to another set. $f: D_f \mapsto R_f$, where $R_f \subseteq CoDomain_f$. Codomain 
is the set where the elements a function can possibly maps to, while range is the set of elements which the function 
actually maps to.
\section{}
The power set of $S$ is a set of all subsets of $S$, which has cardinality $2^{|S|}$ For example $f: \emptyset \mapsto \{\emptyset\}$
\section{}
Base case: 
\[\sum_{k=0}^{0} k^2 = \frac{0(0+1)(2\times0+1)}{6} = 0\]
Inductive case: Suppose 
\[\sum_{k=0}^{n}k^2 = \frac{n(n+1)(2n+1)}{6}\]
\begin{align*}
  \sum_{k=0}^{n+1}k^2 &= \frac{n(n+1)(2n+1)}{6} + (n+1)^2\\
  &= \frac{1}{6}(2n^3 + 3n^2 + n + 6n^2 + 12n + 6)\\
  &= \frac{1}{6}(2n^3 + 9n^2 + 13n + 6)\\
  &= \frac{1}{6}((n+1)(2n^2+7n+6))\\
  &= \frac{1}{6}(n+1)(n+2)(2n+3) 
\end{align*}
\section{}
Base case: $h = 1, |V| = 2^1 - 1 = 1$
Inductive steps: strong induction: \\
Suppose a CBST of height $h$ has $2^h - 1$ nodes, and a CBST with height $h - 1$ has $2^{h - 1} - 1$ nodes, 
The number of leaves $ = 2^{h} - 1 - (2^{h - 1} - 1) = 2^{h - 1}$, to add a new level each leaf grows 2 new leave, total $2 \cdot 2^{h - 1} = 2^h$
Thus a CBST of height $h + 1$ has $2^h - 1 + 2^h = 2^{h+1} - 1$
\section{}
A string $w$ over an alphabet $\Sigma$ is a finitely long sequence of characters from $\Sigma$. A langauge is all strings formed from an alphabet based on certain rules. 
$\Sigma^*$ is a regular language that consists of all finitely long sequence from $\Sigma$.
\section{}
A regular language: the empty language $\emptyset$. proof: trivial, a DFA that accepts nothing.\\
A context free language: valid parentheses, rules:
\begin{align*}
  S &\rightarrow \epsilon\\
  S &\rightarrow SS  \\
  S &\rightarrow (S)
\end{align*}
\section{} 
A turing machine is a state machine defined as such: $\{Q, \Gamma, \Sigma, \delta, b, s, f\}$ each represents:
the set of states, alphabet symbols, tape sympols, transition rules $(Q - F\times \Gamma) \mapsto (Q\times \Gamma \times \{L, R\})$, blank symbol, starting and final states. 
Languages can be either accepted or rejected by a Turing machine is called recursive. Languages that may be accepted but not guarenteed to terminate on a Turing machine is recursively enumerable.
\section{}
Decideable means always terminate of a Turing machine, either accepted or rejected in finite amount of time. For example $\emptyset$, as regular languages are decidable. 
\section{}
Post's correspondence problem is undecidable. 
\section{}
One turing machine terminate on a string in time polynomial wrt input size. Meaning the language is in complexity class P
\section{}
NP complexity class consists of languages that can be accepted or rejected a non-deterministic turing machine in polynomial time wrt input size.

\end{document}
